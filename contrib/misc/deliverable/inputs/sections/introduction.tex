\section{Introduction}

Big data has transformed the way we manage information. As an unprecedented volume of data has become available, there is an increasing demand for stream processing platforms to transform raw data into meaningful knowledge. These velocity-oriented platforms may rely on cloud databases to provide fast data management of continuous and contiguous flows of data with horizontal scalability. Therefore, cloud databases represent an important technology component for a broad range of data-driven domains, including social media, online advertisement, financial trading, security services, and policy-making process.

The architecture of row-store-based relational databases has evolved to meet the requirements of big data on the cloud~\cite{ren2012lightweight}, like elasticity, data partitioning, shared nothing, and especially high performance. The so-called NewSQL databases offer high-speed, scalable data processing in main-memory with consistency guarantees through ACID (atomicity, consistency, isolation, and durability) transactions.

To ensure fast data management, NewSQL databases rely on built-in, fault-tolerance mechanisms, like data partitioning, replication, redundant network topologies, load balancing, and failover. Although these mechanisms handle fail-stop failures successfully, many other cloud performance anomalies may remain unnoticed~\cite{server_delays}. For instance, Do \emph{et al.}~\cite{do2013limplock} found that a single limping network interface can cause a three orders of magnitude execution slowdown in cloud databases. Therefore, we believe that the dependability of NewSQL databases might be improved by detecting these anomalies. 

This paper proposes Tejo, a supervised anomaly detection scheme for NewSQL databases. We make three specific contributions. First, we introduce a scheme for analysing performance anomalies using fault injection tools and a supervised learning model. Second, we shed some light on the impact of performance anomalies in NewSQL databases. Third, we highlight the importance of selecting the proper features and statistical learning algorithm to enhance the anomaly detection efficiency on these databases.

In the next section, we lay out the recent trends in data stream processing and anomaly detection with statistical learning. Following this, in Section 3 we describe the design of Tejo, in particular its components and its two-phased functioning, namely learning and detection phase. In Section 4 we evaluate VoltDB, a prominent NewSQL database, using Tejo. In our experimental setup, VoltDB served two workloads, whose data was partitioned and replicated across a cluster of virtual machines (VMs). Finally, we discuss the related work in Section 5, and conclude in Section 6.


