\section{Conclusion}
\label{sec:conclusion}

The emerging stream processing platforms rely on NewSQL databases deployed on the cloud to compute big data with high velocity. However, performance anomalies caused by faults on the cloud infrastructure, that are likely to be common, may undermine the capacity of NewSQL databases to handle fast data processing. To analyse these performance anomalies, we proposed Tejo, a supervised anomaly detection scheme for NewSQL databases. This scheme allows us to evaluate the performance of NewSQL database as faults on network, memory, CPU, and disk occur. Experiments with VoltDB, a prominent NewSQL database, showed that the 99$^{th}$ percentile latency soars two orders of magnitude as memory and network faults happen. We showed that Tejo also provides a learning model to detect these performance anomalies. Our findings suggest that learning algorithms based on boosting methods are better to detect anomalies on a VoltDB cluster, and features from the TCP layer of VMs are the best predictors. Results also suggest that the contribution of VoltDB-specific features is negligible, therefore our learning model is likely to have similar efficiency with different NewSQL databases.
